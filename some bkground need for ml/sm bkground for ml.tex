\documentclass[UTF8]{ctexart}
\usepackage{amsmath}
\usepackage{amssymb}
\usepackage{bm}
\usepackage{hyperref}
\usepackage{color}


\newtheorem{theorem}{Theorem}

\title{some background need for ml}
\author{zhangyifeng}
\begin{document}
\maketitle
\tableofcontents
\newpage
\section{Matrix}
	\subsection{notation}
		\noindent$\bm \alpha \in \mathbb{R}^{n}$\\
		$\bm x \in \mathbb{R}^{n}$\\
		$A\in \mathbb{R}^{m\times n}$\\\
		$(A)_{ij}=a_{ij}$\\
		$A^{T}$: transpose of $A$\\
		$tr(A)=\sum_{i=1}^{n}a_{ii}$\\
		$det(A)=\sum_{\sigma\in S_{n}}par(\sigma)a_{1\sigma_{1}}a_{2\sigma_{2}}\dots_{1\sigma_{n}}$, where $S_{n}$ is all set of $n-order$ permutation. par($\sigma$) can be -1 or +1.\\
		also, $det(A)=\sum_{i=1}^{n}a_{ki}A_{ki} (k=1,2,\dots,n)=\sum_{j=1}^{n}a_{jl}A_{jl} (l=1,2,\dots,n)$\\
		Frobenius norm of $A$:
		$$\|A\|_{F}=(tr(A^{T}A))^{1/2}=(\sum\limits_{i=1}^{m}\sum\limits_{j=1}^{n}a_{ij}^{2})^{1/2}$$
		it can be regarded as $L_{2}$ norm when metrix was extended to vectors
	\subsection{Basic calculate}
		\noindent$tr(AB)=tr(BA)$\\
		$tr(ABC)=tr(BCA)=tr(CAB)$





\section{common distribution}
	\subsection{gamma distribution}
		\subsubsection*{gamma function}
			$\Gamma(a)=\int_{0}^{\infty}x^{a-1}\exp^{-x}dx$, where $a>0$

\section{Matrix Differentiation}
	\subsection{Matrix Differentiation-from functional analysis points}
		假设$X$和$Y$为赋范向量空间,$F: X\rightarrow Y$是一个映射,那么$F$在$x_0 \in X$可导的意思是说存在一个有界线性算子$L \in \mathcal{L}(X, Y)$,使得对于任意的$\epsilon > 0$都存在$\delta > 0$,对于满足$x \in X \backslash \{x_0\}, \|x - x_0\| < \delta$的$x$都有$\frac{\|F(x) - F(x_0) - L(x - x_0)\|}{\|x - x_0\|} < \epsilon$.我们称$L(\|x - x_0\|)$为$F$在$x_0$点的微分。
		
		以上定义有一个等价的表述,往往计算起来更方便:对于距离$x_0$足够近的点$x$,即
		$\lim_{x \rightarrow x_0}\frac{o(\|x-x_0\|)}{\|x-x_0\|} = 0$,
		有$F(x) = F(x_0) + L(x - x_0) + o(\|x - x_0\|)$.
		(注:此处$L(x-x_0)$应该理解为线性算子$L$在$x - x_0$这个点的值,而不是$L$乘以$x-x_0$。不过在有限维空间所有线性算子都可以用矩阵表述,$L$在$x - x_0$,这个值便正好可以表述为矩阵与向量的乘积(Riesz表示定理))

		例子1:假设$F(X) = X^TX$是一个$\mathbb{R}^{m\times n} \rightarrow \mathbb{ S}^n$的映射,其中$\mathbb{S }^n$为$n$维对称阵的空间。
		\begin{align*} 
		&F(X+\Delta X) - F(X) \\ 
		=& (X+\Delta X)^T(X+\Delta X) - X^TX \\ 
		=& X^T\Delta X + \Delta X^TX + o(\|\Delta X\|) \\
		% =& 2X^T\Delta X + o(\|\Delta X\|) 
		\end{align*}
		所以我们有$L(\Delta X) = 2X^T\Delta X$,这个就是$F$在$X$点的微分。

		例子2:最小二乘问题$f(x) = \frac{1}{2}\|Ax-b\|_2^2,f$是一个$\mathbb R^n \rightarrow \mathbb R$的映射。

		\begin{align*}
		 &f(x+\Delta x) - f(x) \\ 
		 =& \frac{1}{2}\|A(x+\Delta x) - b\|^2 - \frac{1}{2}\|Ax - b\|^2\\ 
		 =& \frac{1}{2}\|Ax - b + A\Delta x\|^2 - \frac{1}{2}\|Ax - b\|^2\\
		 =& (Ax - b)^TA\Delta x + o(\|\Delta x\|) 
		\end{align*}

		所以我们有$L(\Delta x) = (Ax - b)^TA\Delta x$,这个就是$f$在$x$点的微分。在这种情况下,$L$这个有界线性算子(梯度)可以用矩阵来表述(Riesz表示定理):
		$L(\Delta x) = \langle \nabla f(x), \Delta x \rangle =  (Ax-b)^TA\Delta x$,
		所以梯度$\nabla f(x) = A^T(Ax - b)$

		总结:在有限维的情况下,我们可以先求$F$的微分$L(\Delta x)$,利用Riesz表示定理,得$L(\Delta x) = \langle f^{'}(x) , \Delta x \rangle$,可求得对应的gradient vector或者jacobi矩阵$f^{'}(x)$,也就是导数,显然, 这里可以看出,导数和微分差一个转置。

	\subsection*{标量$f$对矩阵$X$的导数}
		\subsubsection*{核心思想}
		函数的微分 = 函数的导数 和 自变量的微分 的内积 $= \text{tr}\left(\frac{\partial f}{\partial X}^T dX\right)$
		\subsubsection*{矩阵微分运算法则}
		加减法:$d(X\pm Y) = dX \pm dY$

		矩阵乘法:$d(XY) = (dX)Y + X dY$ 

		转置:$d(X^T) = (dX)^T$

		迹:$d\text{tr}(X) = \text{tr}(dX)$

		逆:$dX^{-1} = -X^{-1}dX X^{-1}$。此式可在$XX^{-1}=I$两侧求微分来证明

		行列式:$d|X| = \text{tr}(X^{\#}dX)$ ,其中$X^{\#}$表示X的伴随矩阵,在X可逆时又可以写作$d|X|= |X|\text{tr}(X^{-1}dX)$。此式可用Laplace展开来证明,详见张贤达《矩阵分析与应用》第279页

		逐元素乘法:$d(X\odot Y) = dX\odot Y + X\odot dY,\odot$表示尺寸相同的矩阵X,Y逐元素相乘

		逐元素函数:$d\sigma(X) = \sigma'(X)\odot dX$ ,$\sigma(X) = \left[\sigma(X_{ij})\right]$是逐元素标量函数运算, $\sigma'(X)=[\sigma'(X_{ij})]$是逐元素求导数。

		举个例子,
		$X =
		\left[\begin{matrix}
		x_{11} & x_{12} \\
		x_{21} & x_{22}
		\end{matrix}\right], 
		d \sin(X) = 
		\left[\begin{matrix}\cos x_{11} dx_{11} & \cos x_{12} d x_{12}\\ 
		\cos x_{21} d x_{21}& \cos x_{22} dx_{22}
		\end{matrix}\right] =
		\cos(X)\odot dX$。


		\subsubsection*{迹技巧(trace trick)}
		标量套上迹:$a = \text{tr}(a)$

		转置:${tr}(A^T) = {tr}(A)$

		线性:$\text{tr}(A\pm B) = \text{tr}(A)\pm \text{tr}(B)$

		矩阵乘法交换:$\text{tr}(AB) = \text{tr}(BA)$,其中$A$与$B^T$尺寸相同。两侧都等于$\sum_{i,j}A_{ij}B_{ji}$

		矩阵乘法/逐元素乘法交换:$\text{tr}(A^T(B\odot C)) = \text{tr}((A\odot B)^TC)$,其中$A$, $B$, $C$尺寸相同。两侧都等于$\sum_{i,j}A_{ij}B_{ij}C_{ij}$

		\subsubsection*{复合法则}
		假设已求得$\frac{\partial f}{\partial Y}$,而$Y$是$X$的函数,如何求$\frac{\partial f}{\partial X}$呢?在微积分中有标量求导的链式法则$\frac{\partial f}{\partial x} = \frac{\partial f}{\partial y} \frac{\partial y}{\partial x}$,但这里我们不能沿用链式法则,因为矩阵对矩阵的导数$\frac{\partial Y}{\partial X}$截至目前仍是未定义的。于是我们继续追本溯源,链式法则是从何而来?源头仍然是微分。我们直接从微分入手建立复合法则:先写出$df = \text{tr}\left(\frac{\partial f}{\partial Y}^T dY\right)$,再将$dY$用$dX$表示出来代入(这个是矩阵对矩阵的导数,在下一节我们会了解到),并使用迹技巧将其他项交换至dX左侧,即可得到$\frac{\partial f}{\partial X}$。

		\subsubsection*{标量对矩阵的一般求导步骤}
		{\color{red}

		1.对标量函数$f$两端作微分,利用微分运算法则化简

		2.对两端作迹运算,利用迹运算法则化简,将$dx$移到最右端

		3.利用微分和矩阵的联系$df = \text{tr}\left(\frac{\partial f}{\partial X}^T dX\right)$,求$\frac{\partial f}{\partial X}$}

		\subsubsection*{一些例子}
		\noindent 例1:$f = \boldsymbol{a}^T X\boldsymbol{b}$,求$\frac{\partial f}{\partial X}$。其中$\boldsymbol{a}$是$m×1$列向量,$X$是$m\times n$矩阵,$\boldsymbol{b}$是$n×1$列向量,$f$是标量。

		\noindent 解:
		1.作微分:这里的$\boldsymbol{a}, \boldsymbol{b}$是常量,$d\boldsymbol{a} = \boldsymbol{0}$, $d\boldsymbol{b} = \boldsymbol{0}$,得:$df = \boldsymbol{a}^T dX\boldsymbol{b}$ 

		2.作迹运算:$df = \text{tr}(\boldsymbol{a}^TdX\boldsymbol{b}) = \text{tr}(\boldsymbol{b}\boldsymbol{a}^TdX)$,注意这里我们根据$\text{tr}(AB) = \text{tr}(BA)交换了\boldsymbol{a}^TdX与\boldsymbol{b}$

		3.对照导数与微分的联系$df = \text{tr}\left(\frac{\partial f}{\partial X}^T dX\right)$,得到$\frac{\partial f}{\partial X} = (\boldsymbol{b}\boldsymbol{a}^T)^T= \boldsymbol{a}\boldsymbol{b}^T$。\\

		\noindent 例2:$f = \boldsymbol{a}^T \exp(X\boldsymbol{b})$,求$\frac{\partial f}{\partial X}$。其中$\boldsymbol{a}$是$m×1$列向量,$X$是$m\times n$矩阵,$\boldsymbol{b}$是$n×1$列向量,$exp$表示逐元素求指数,$f$是标量。

		\noindent 解:
		1.作微分:$df = \boldsymbol{a}^T(\exp(X\boldsymbol{b})\odot (dX\boldsymbol{b}))$

		2.作迹运算:$df = \text{tr}( \boldsymbol{a}^T(\exp(X\boldsymbol{b})\odot (dX\boldsymbol{b}))) =\text{tr}((\boldsymbol{a}\odot \exp(X\boldsymbol{b}))^TdX \boldsymbol{b}) = \text{tr}(\boldsymbol{b}(\boldsymbol{a}\odot \exp(X\boldsymbol{b}))^TdX)$

		3.对照导数与微分的联系$df = \text{tr}\left(\frac{\partial f}{\partial X}^T dX\right)$,得到$\frac{\partial f}{\partial X} = (\boldsymbol{b}(\boldsymbol{a}\odot \exp(X\boldsymbol{b}))^T)^T= (\boldsymbol{a}\odot \exp(X\boldsymbol{b}))\boldsymbol{b}^T$。\\

		\noindent 例3【线性回归】:$l = \|X\boldsymbol{w}- \boldsymbol{y}\|^2$, 求$\boldsymbol{w}$的最小二乘估计,即求$\frac{\partial l}{\partial \boldsymbol{w}}$的零点。其中$\boldsymbol{y}$是$m×1$列向量,$X$是$m\times n$矩阵,$\boldsymbol{w}$是$n×1$列向量,$l$是标量。

		\noindent 解:严格来说这是标量对向量的导数,不过可以把向量看做矩阵的特例(此时可以省略第二步:作迹运算)。

		先将向量模平方改写成向量与自身的内积:$l = (X\boldsymbol{w}- \boldsymbol{y})^T(X\boldsymbol{w}- \boldsymbol{y})$

		1.求微分:$dl = (Xd\boldsymbol{w})^T(X\boldsymbol{w}-\boldsymbol{y})+(X\boldsymbol{w}-\boldsymbol{y})^T(Xd\boldsymbol{w}) = 2(X\boldsymbol{w}-\boldsymbol{y})^TXd\boldsymbol{w}$。

		2.对照导数与微分的联系$dl = \frac{\partial l}{\partial \boldsymbol{w}}^Td\boldsymbol{w}$,得到$\frac{\partial l}{\partial \boldsymbol{w}}= (2(X\boldsymbol{w}-\boldsymbol{y})^TX)^T = 2X^T(X\boldsymbol{w}-\boldsymbol{y})$。$\frac{\partial l}{\partial \boldsymbol{w}}$的零点即$\boldsymbol{w}$的最小二乘估计为$\boldsymbol{w} = (X^TX)^{-1}X^T\boldsymbol{y}$。\\

		\noindent 例4【方差的最大似然估计】:样本$\boldsymbol{x}_1,\dots, \boldsymbol{x}_n\sim N(\boldsymbol{\mu}, \Sigma)$,求方差$\Sigma$的最大似然估计。写成数学式是:$l = \log|\Sigma|+\frac{1}{n}\sum_{i=1}^n(\boldsymbol{x}_i-\boldsymbol{\bar{x}})^T\Sigma^{-1}(\boldsymbol{x}_i-\boldsymbol{\bar{x}})$,求$\frac{\partial l }{\partial \Sigma}$的零点。其中$\boldsymbol{x}_i$是$m\times 1$列向量,$\overline{\boldsymbol{x}}=\frac{1}{n}\sum_{i=1}^n \boldsymbol{x}_i$是样本均值,$\Sigma$是$m\times m$对称正定矩阵,$l$是标量。

		\noindent 解:
		1.作微分:第一项是$d\log|\Sigma| = |\Sigma|^{-1}d|\Sigma| = \text{tr}(\Sigma^{-1}d\Sigma)$,第二项是$\frac{1}{n}\sum_{i=1}^n(\boldsymbol{x}_i-\boldsymbol{\bar{x}})^Td\Sigma^{-1}(\boldsymbol{x}_i-\boldsymbol{\bar{x}}) = -\frac{1}{n}\sum_{i=1}^n(\boldsymbol{x}_i-\boldsymbol{\bar{x}})^T\Sigma^{-1}d\Sigma\Sigma^{-1}(\boldsymbol{x}_i-\boldsymbol{\bar{x}})$。

		2.作迹运算:$\text{tr}\left(\frac{1}{n}\sum_{i=1}^n(\boldsymbol{x}_i-\boldsymbol{\bar{x}})^T\Sigma^{-1}d\Sigma\Sigma^{-1}(\boldsymbol{x}_i-\boldsymbol{\bar{x}})\right) = \frac{1}{n} \sum_{i=1}^n \text{tr}((\boldsymbol{x}_i-\boldsymbol{\bar{x}})^T\Sigma^{-1} d\Sigma \Sigma^{-1}(\boldsymbol{x}_i-\boldsymbol{\bar{x}}))= \frac{1}{n}\sum_{i=1}^n\text{tr}\left(\Sigma^{-1}(\boldsymbol{x}_i-\boldsymbol{\bar{x}})(\boldsymbol{x}_i-\boldsymbol{\bar{x}})^T\Sigma^{-1}d\Sigma\right)$\\
		$=\text{tr}(\Sigma^{-1}S\Sigma^{-1}d\Sigma)$
		,定义$S = \frac{1}{n}\sum_{i=1}^n(\boldsymbol{x}_i-\boldsymbol{\bar{x}})(\boldsymbol{x}_i-\boldsymbol{\bar{x}})^T$为样本方差矩阵。得到$dl = \text{tr}\left(\left(\Sigma^{-1}-\Sigma^{-1}S\Sigma^{-1}\right)d\Sigma\right)$。

		3.对照导数与微分的联系,有$\frac{\partial l }{\partial \Sigma}=(\Sigma^{-1}-\Sigma^{-1}S\Sigma^{-1})^T$,其零点即$\Sigma$的最大似然估计为$\Sigma = S$。

	\subsection*{矩阵$F$对矩阵$X$的导数}
		一般而言,标量就是$1\times1$的矩阵,如果我们能推导出矩阵对矩阵的导数,标量对矩阵的导数不是自然的么,不应该可以统一进来么,那为啥还要大费周章地先写标量对矩阵的导数。原因是这两者不完全相同,并不能很简单地统一起来。

		应该怎么定义矩阵对矩阵的导数。回答这个问题不是随意的,为了满足两个要求,我们对矩阵对矩阵的定义有严格的要求。我们的两个要求是:

		1.矩阵$F\in \mathbb{R}^{p \times q}$对矩阵$X\in \mathbb{R}^{m \times n}$的导数应包含所有$mnpq$个偏导数$\frac{\partial F_{kl}}{\partial X_{ij}}$,从而不损失信息。

		2.在标量对矩阵求导的地方,我们发现导数与微分有简明的联系。这里我们仍希望他们之间存在某种联系。

		为此,我们先定义向量$\boldsymbol{f}(p×1)$对向量$\boldsymbol{x}(m×1)$的导数
		$$\frac{\partial \boldsymbol{f}}{\partial \boldsymbol{x}} 
		= 
		\begin{bmatrix} 
		\frac{\partial f_1}{\partial x_1} & \frac{\partial f_2}{\partial x_1} & \cdots & \frac{\partial f_p}{\partial x_1}\\ 
		\frac{\partial f_1}{\partial x_2} & \frac{\partial f_2}{\partial x_2} & \cdots & \frac{\partial f_p}{\partial x_2}\\ 
		\vdots & \vdots & \ddots & \vdots\\ 
		\frac{\partial f_1}{\partial x_m} & \frac{\partial f_2}{\partial x_m} & \cdots & \frac{\partial f_p}{\partial x_m}\\ 
		\end{bmatrix}(m×p)$$
		此时,可以证明,$d\boldsymbol{f} = \sum\limits_{i,j}\frac{\partial f_{i}}{\partial x_{j}} dx_{j} =
		\frac{\partial \boldsymbol{f} }{\partial \boldsymbol{x} }^T d\boldsymbol{x}$,这个定义满足我们的两个要求,所以我们现在作好了了向量对向量的导数。

		再定义矩阵的(按列优先)向量化:
		$${vec}(X) = [X_{11}, \ldots, X_{m1}, X_{12}, \ldots, X_{m2}, \ldots, X_{1n}, \ldots, X_{mn}]^T(mn×1)$$
		并定义矩阵F对矩阵X的导数$\frac{\partial F}{\partial X} $= $\frac{\partial {vec}(F)}{\partial {vec}(X)}(mn×pq)$。
		此时,可以证明,导数与微分有联系${vec}(dF)$ = $\frac{\partial F}{\partial X}^T {vec}(dX)$,这样,我们作好了满足要求的矩阵关于矩阵的导数。

		\subsubsection*{列向量化运算法则}

		1.线性:${vec}(A+B) = {vec}(A) + {vec}(B)$。
		
		2.{\color{red} 矩阵乘法}:${vec}(AXB) = (B^T \otimes A) {vec}(X)$,其中$\otimes$表示Kronecker积,$A(m×n)与B(p×q)$的Kronecker积是$A\otimes B = [A_{ij}B](mp×nq)$。此式证明见张贤达《矩阵分析与应用》第107-108页。
		
		3.转置:${vec}(A^T) = K_{mn}{vec}(A)$,$A$是$m×n$矩阵,其中$K_{mn}(mn×mn)$是换位矩阵(commutation matrix)(就是一些初等换位矩阵的乘积)。

		4.逐元素乘法:${vec}(A\odot X) = {diag}(A){vec}(X)$,其中${diag}(A)(mn×mn)$是用$A$的元素(按列优先)排成的对角阵。

		\subsubsection*{一些Kronecker积和交换矩阵相关的恒等式}

		1.$(A\otimes B)^T = A^T \otimes B^T$。
		
		2.${vec}(\boldsymbol{ab}^T) = \boldsymbol{b}\otimes\boldsymbol{a}$。

		3.$(A\otimes B)(C\otimes D) = (AC)\otimes (BD)$。可以对$F = D^TB^TXAC$求导来证明,一方面,直接求导得到$\frac{\partial F}{\partial X} = (AC) \otimes (BD)$;另一方面,引入$Y = B^T X A$,有$\frac{\partial F}{\partial Y} = C \otimes D$, $\frac{\partial Y}{\partial X} = A \otimes B$,用链式法则得到$\frac{\partial F}{\partial X} = (A\otimes B)(C \otimes D)$。
		
		4.$K_{mn} = K_{nm}^T, K_{mn}K_{nm} = I$,所以换位矩阵是正交矩阵。

		5.$K_{pm}(A\otimes B) K_{nq} = B\otimes A$,$A$是$m×n$矩阵,$B$是$p×q$矩阵。可以对$AXB^T$做向量化来证明,一方面,${vec}(AXB^T) = (B\otimes A){vec}(X)$;另一方面,${vec}(AXB^T) = K_{pm}{vec}(BX^TA^T) = K_{pm}(A\otimes B){vec}(X^T) = K_{pm}(A\otimes B) K_{nq}{vec}(X)$。

		\subsubsection*{复合法则}
		假设已求得$\frac{\partial F}{\partial Y}$,而$Y$是$X$的函数,如何求$\frac{\partial F}{\partial X}$呢?从导数与微分的联系入手,${vec}(dF) = \frac{\partial F}{\partial Y}^T{vec}(dY) = \frac{\partial F}{\partial Y}^T\frac{\partial Y}{\partial X}^T{vec}(dX) $,可以推出链式法则$\frac{\partial F}{\partial X} = \frac{\partial Y}{\partial X}\frac{\partial F}{\partial Y}$

		\subsubsection*{矩阵对矩阵的一般求导步骤}
		{\color{red}
		1.对矩阵值函数$F$两端作微分,利用微分运算法则化简

		2.对两端作列向量化运算,利用列向量化法则化简,注意看列向量里面是什么形式,就用什么公式,如列向量里面是两个矩阵相乘,就想办法凑进去一个单位矩阵,并使得${vec}x$在中间,然后利用vec的矩阵乘法公式

		3.利用微分和矩阵的联系${vec}(dF)$ = $\frac{\partial F}{\partial X}^T {vec}(dX)$,求$\frac{\partial f}{\partial X}$}

		\subsubsection*{一些例子}
		\noindent 例1:$F = AX$,$X$是$m×n$矩阵,求$\frac{\partial F}{\partial X}$。

		\noindent 解:
		1.作微分:$dF=AdX$

		2.列向量化,使用矩阵乘法的技巧,注意在$dX$右侧添加单位阵:${vec}(dF) = {vec}(AdX) = (I_n\otimes A){vec}(dX)$

		3.对照导数与微分的联系得到$\frac{\partial F}{\partial X} = I_n\otimes A^T$。

		特例:如果$X$退化为向量,即$\boldsymbol{f} = A \boldsymbol{x}$,则根据向量的导数与微分的关系d$\boldsymbol{f} = \frac{\partial \boldsymbol{f}}{\partial \boldsymbol{x}}^T d\boldsymbol{x}$,得到$\frac{\partial \boldsymbol{f}}{\partial \boldsymbol{x}} = A^T$

		$df(\mathbf{X},\mathbf{Y}) = tr(\frac{\partial f}{\partial \mathbf{X}}^T d\mathbf{X}) + tr(\frac{\partial f}{\partial \mathbf{Y}}^T d\mathbf{Y})$\\

		\noindent 例2:$f = \log |X|$ ,X是n×n矩阵,求$\nabla^2_X f$。

		\noindent 解:
		1.求微分:$d\nabla_X f = -(X^{-1}dXX^{-1})^T$

		2.列向量化,${vec}(d\nabla_X f)= -K_{nn}{vec}(X^{-1}dX X^{-1}) = -K_{nn}(X^{-T}\otimes X^{-1}){vec}(dX)$,

		3.对照导数与微分的联系,得到$\nabla^2_X f = -K_{nn}(X^{-T}\otimes X^{-1})$,注意它是对称矩阵。\\


		\noindent 例3:$F = A\exp(XB),A$是$l×m$矩阵,$X$是$m×n$矩阵,$B$是$n×p$矩阵,$exp$为逐元素函数,求$\frac{\partial F}{\partial X}$。

		\noindent 解:
		1.求微分:$dF = A(\exp(XB)\odot (dXB))$

		2.列向量化:${vec}(dF) = (I_p\otimes A){vec}(\exp(XB)\odot (dXB)) = \\(I_p \otimes A) {diag}(\exp(XB)){vec}(dXB) = (I_p\otimes A){diag}(\exp(XB))(B^T\otimes I_m){vec}(dX)$。

		3.对照导数与微分的联系得到$\frac{\partial F}{\partial X} = (B\otimes I_m){diag}(\exp(XB))(I_p\otimes A^T)$。

		\subsubsection*{注解}

		1.一般而言,这套方法就是为了矩阵对矩阵求导而引入的,由于这里是利用列向量定义的导数,所以直接应用在标量对矩阵$X\in \mathbb{R}^{m\times n}$的导数上,会得到一个$mn\times1$的列向量,这与我们一般定义的标量对矩阵的导数相悖,所以一般标量对矩阵的导数,我们还是利用上一节的方法。当然,若将上一节定义的标量$f(X)\in \mathbb{R}^{1}$对矩阵$X\in \mathbb{R}^{m\times n}$的导数用记号$\nabla_X f\in \mathbb{R}^{m\times n}$来表示,则这里定义的$\frac{\partial f}{\partial X}={vec}(\nabla_X f)$,在牢记这一条的情况下,我们可以用本节的方法来解决标量对矩阵求导,只是没有上一节的方法方便。为了满足读者的好奇心,我们给出标量对矩阵求导的一个例子,并且用两种方法来解决。

		2.标量对矩阵的二阶导数,又称Hessian矩阵,定义为$\nabla^2_X f = \frac{\partial^2 f}{\partial X^2} = \frac{\partial \nabla_X f}{\partial X}(mn×mn)$是对称矩阵,这个二阶导数分两次进行,第一次是标量对矩阵求导,第二次是矩阵对矩阵求导。

		3.如何理解$K_{mn}(mn×mn)$,它是一个换位矩阵,根据${vec}(A^T) = K_{mn}\\{vec}(A)$,它的作用是使的${vec}(A^T)$和${vec}(A)$的若干行对换位置。由$[A]_{i,j}=[A]_{j,i}=[{vec}(A^T)]_{(i-1)n+j}=[{vec}(A)]_{(j-1)n+i}$, 这里$A\in \mathbb{R}^{m\times n}, 1 \leq i \leq m, 1 \leq j \leq n$,所以$K_{mn}$就是单位矩阵(mn×mn)交换$(i-1)n+j$和$(j-1)n+i$行得到的一个矩阵。

		\subsection*{对两节内容的总结}
		我们发展了从整体出发的矩阵求导的技术,导数与微分的联系是计算的枢纽。

		上一节中,我们了解了,标量对矩阵的导数与微分的联系是$df =\\ {tr}((\nabla_Xf)^T dX)$,先对$f$求微分,再使用迹技巧可求得导数,特别地,标量对向量的导数与微分的联系是$df = (\nabla_{\boldsymbol{x}}f)^T d\boldsymbol{x}$

		下一节中,我们了解了,矩阵对矩阵的导数与微分的联系是${vec}(dF) = \frac{\partial F}{\partial X}^T {vec}(dX)$,先对$F$求微分,再使用列向量化的技巧可求得导数,特别地,向量对向量的导数与微分的联系是$d\boldsymbol{f} = \frac{\partial \boldsymbol{f}}{\partial \boldsymbol{x}}^Td\boldsymbol{x}$。

	\subsection*{reference}
	\href{https://www.zhihu.com/question/39523290}{如何理解矩阵对矩阵求导?-知乎-猪猪专业户}

	\href{https://zhuanlan.zhihu.com/p/24709748}{矩阵求导术(上)-知乎-长躯鬼侠}

	\href{https://zhuanlan.zhihu.com/p/24863977}{矩阵求导术(下)-知乎-长躯鬼侠}


\section{Lagrange duality}
	\subsection{application}
		applied on:\\
		- 最大熵模型\\
		- SVM(support vector machine)
	\subsection{primal problem}
		Set $f(\bm x),c_{i}(\bm x),h_{j}(\bm x)$ are continuously differentiable function over $bm R{n}$, consider optimization problem with constraints
		\begin{equation*}
		\begin{split}
		&\min_{\bm x\in \bm R^{n}}\ f(\bm x) \\
		s.t.\  c_{i}(\bm x)&\leq0, \ i=1,2,\cdots,k\\
		h_{j}(\bm x)&=0, \ j=1,2,\cdots,l
		\end{split}
		\end{equation*}
	\subsection{generalized Lagrange function}
		$$
		L(\bm x,\bm\alpha,\bm\beta)=f(\bm x)+\sum\limits_{i=1}^{k}a_{i}c_{i}(\bm x)+\sum\limits_{j=1}^{l}\beta_{j}h_{j}(\bm x)
		$$
		where, $\bm x=(x^{1},x^{2},\dots,x^{n})^{T}\in\bm R^{n}$,$\alpha_{i},\beta_{j}$ are Lagrange multiplier, $\alpha_{i}\geq0$ \\
		After introduced generalized Lagrange function, primal problem is equal to 
		\begin{equation*}
		\begin{split}
		&\min\limits_{\bm x}\max\limits_{\bm \alpha, \bm \beta}L(\bm x,\bm\alpha,\bm\beta)\\
		&s.t.\ \alpha_{i}\geq0, \ i=1,2,\dots,k
		\end{split}
		\end{equation*}

	\subsection{dual problem}
		\begin{equation*}
		\begin{split}
		&\max\limits_{\bm\alpha,\bm\beta}\min\limits_{\bm x}L(\bm x,\bm\alpha,\bm\beta)\\
		&s.t.\ \alpha_{i}\geq0, \ i=1,2,\dots,k
		\end{split}
		\end{equation*}
	\subsection{KKT(Karush-Kuhn-Tucker)condition}
		\begin{equation*}
		\begin{split}
		\nabla_{x}L(\bm x,\bm\alpha,\bm\beta)&=0\\
		\nabla_{\alpha}L(\bm x,\bm\alpha,\bm\beta)&=0\\
		\nabla_{\beta}L(\bm x,\bm\alpha,\bm\beta)&=0\\
		\alpha_{i}c_{i}(\bm x)=0, \ i=&1,2,\dots,k\\
		c_{i}(\bm x)\leq0, \ i=&1,2,\dots,k\\
		\alpha_{i}\geq0, i=&1,2,\dots,k\\
		h_{j}(\bm x)=0, j=&1,2\dots,l
		\end{split}
		\end{equation*}

		\begin{theorem}
		if $f(\bm x)$ and $c_{i}(\bm x)$ are convex function, $h_{j}(\bm x)$ are affine function\footnote{$f(x)$ is called affine function, when it holds that $f(x)=\bm a\cdot \bm x+b, \bm a\in \bm{R}^{n},b\in \bm R, \bm x\in R^{n}$}, and inequation constrains $c_{i}(\bm x)$ strictly hold, that is, exist $\bm x$, s.t. for any $i$, hold $c_{i}(\bm x)<0$, then, there must be $\bm x^{*},\bm\alpha^{*},\bm\beta^{*}$ are the optimal solution of primal problem as well as dual problem and satisfy KKT condition at $\bm x^{*},\bm\alpha^{*},\bm\beta^{*}$.
		\end{theorem}
		Remark: so, when the prerequisites are satisfied, we can use KKT condition to find the optimal solution $\bm x^{*},\bm\alpha^{*},\bm\beta^{*}$.


\end{document}

\begin{thebibliography}{1}
\bibitem{zhou} 周志华. 机器学习 [M]. 北京: 清华大学出版社, 2016.
\end{thebibliography}